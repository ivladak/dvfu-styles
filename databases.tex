\documentclass[a4paper]{article}
\usepackage{dvfu}
\begin{document}

\renewcommand{\CourseName}{Базы данных}
\renewcommand{\CourseAuthor}{старший преподаватель кафедры информатики, математического и компьютерного моделирования А.~С.~Кленин}
\renewcommand{\CourseYear}{2}
\renewcommand{\CourseSemester}{4}
\renewcommand{\CourseLectures}{36}

\renewcommand{\CourseTotalAuditory}{70}
\renewcommand{\CourseHomework}{50}

\renewcommand{\CourseReferatsNumber}{1}

\renewcommand{\CoursePassfailSemester}{3}
\renewcommand{\CourseExamSemester}{2}

\CoursePlanTitlePage
\begin{CoursePlanTitlePageReverse}
  \CourseUpdateRecord
  \CourseUpdateRecord
\end{CoursePlanTitlePageReverse}

\CourseAnnotationSection

Дисциплина относится к обязательной части цикла математического и естественнонаучного цикла
основной образовательной программы подготовки бакалавров на направлении подготовки 010400.62 «Прикладная математика и информатика».

Дисциплина базируется на следующих дисциплинах: «Практикум на ЭВМ», «Основы информатики», «Основы информатики 2».

Знания, полученные по освоении дисциплины, необходимы при выполнении предквалификационной практики и выпускной квалификационной работы.

\CourseGoal

Дать студентам знания о теоретических основах СУБД, включая реляционную теорию, элементы теории распределённых систем,
а также умения настраивать и поддерживать СУБД.
Уделяется внимание глубокому изучению практических аспектов программирования прикладных приложений,
использующих реляционные СУБД, включая технологии ORM.

\CourseTasks
\begin{itemize}
  \item познакомить студентов с теоретическими основами СУБД;
  \item научить студентов устанавливать, настраивать и сопровождать СУБД;
  \item научить студентов проектировать БД;
  \item научить студентов разрабатывать прикладные программы, использующие БД.
\end{itemize}

\CourseTheorySection

1.	История и основные понятия СУБД.
	Понятие базы данных и системы управления базами данных. Основные функции СУБД. Модели баз данных: неструктурированная, иерархическая, сетевая, реляционная. История развития СУБД. Обзор рынка СУБД.
	Основные коммерческие СУБД: Oracle, Microsoft SQL Server.
	Основные открытые реляционные СУБД: Maria DB, PostgreSQL, Firebird, SQLite.
	Основные открытые нереляционные СУБД..
2.	Язык SQL.
	Назначение и история развития языка SQL.
	Стандарты SQL, совместимость различных СУБД.
	Типы данных, операции и функции в SQL.
	Основные объекты СУБД: таблицы, поля, ограничения, домены, триггеры, хранимые процедуры и функции, внешние процедуры и функции, пользователи и роли.
	Понятие метаданных, Data Definition Language (DDL), операторы CREATE, DROP, ALTER.
	Data Manipulation Language (DML), операторы INSERT, UPDATE, DELETE.
	Оператор SELECT, предложения FROM, WHERE, ORDER BY, ROWS. Соединения, INNER JOIN, LEFT/RIGHT JOIN, FULL OUTER JOIN. Группировка, агрегатные функции, предложения GROUP BY, HAVING.
	Прочие языки запросов: XQuery / XPath, DMX, LINQ.
3.	Разработка прикладных программ, работающих с БД.
	Клиент-серверная архитектура, многозвенная архитектура. Протоколы СУБД, требования и особенности. Распределение функций между сервером СУБД и клиентской библиотекой СУБД. Совместимость СУБД, общие протоколы, ODBC, JDBC, стандарты Microsoft.
	Библиотеки доступа к БД из языков программирования высокого уровня. Middleware. Уровни абстрагирования библиотек: универсальное подключение, конструирование запросов, объектно-реляционное отображение (ORM), фреймворки. Достоинства и недостатки ORM, примеры библиотек ORM.
	Этапы обработки SQL-запроса. Параметры запросов. Атака SQL Injection и защита от неё. Оптимизация повторяющихся запросов с помощью предварительной подготовки..
4.	Многопользовательский доступ.
	Понятие целостности данных, виды нарушения целостности.
	Понятие транзакции, свойства ACID.
	Уровни изоляции, поддержка в различных СУБД, влияние уровня изоляции на производительность, критерии выбора уровня изоляции.
	Методы реализаций механизма транзакций, блокировка, версионирование. Обработка конфликтов.
5.	Теория и проектирование реляционных БД.
	Формальные понятия отношения, кортежа, элемента данных. Основные операции реляционной алгебры.
	Понятия зависимости, ключа, кандидата в ключи, суперключа, первичного и внешнего ключа, суррогатного ключа.
	Нормализация, нормальные формы, первая, вторая, третья, четвертая, пятая нормальные формы. Прочие нормальные формы.
	Преимущества и недостатки нормализации. Денормализация, причины и методы.
	Проектирование баз данных. Цели и задачи проектирования. Проектирование БД как инструмент анализа предметной области.
	Визуальное проектирование БД. Программные средства визуального проектирования. Языки проектирования БД: UML, IDEFx.
6.	Аналитические возможности СУБД.
	Аналитические возможности СУБД.  Многомерная модель. Понятия OLAP и OLTP. Аналитические расширения языка SQL: CTE, оконные функции.
	Требования к пользовательскому интерфейсу OLAP-систем, библиотеки и инструментальные средства для аналитических запросов.
7.	Распределённые базы данных.
	Понятие распределённой базы, причины, преимущества и недостатки распределения, требования к распределённым БД. Проблемы реализации распределённых транзакций, теорема CAP.
	Механизмы синхронизации распределённых БД, двух-фазная блокировка (2PL, S2PL, SS2PL), двух-фазное завершение (2PC).
	Репликация, режимы репликации. Проблемы суррогатных ключей при двусторонней репликации, использование GUID.
8.	Администрирование баз данных
	Задачи и инструменты администрирования. Контроль прав доступа, распределение ролей. Мониторинг БД.
	Модели рисков для БД, борьба с рисками. Резервное копирование.
9.	Оптимизация баз данных
	Средства оценки и мониторинга производительности. Основные источники проблем с производительностью, принципы оптимизации.
	Оптимизация запросов. Индексы, виды индексов. План исполнения запросов, его анализ и изменение.
	Обзор алгоритмов и структур данных, используемых при реализации СУБД.

\CoursePracticeSection

1.	Установка и настройка СУБД, установка клиентского приложения
2.	Построение SQL-запросов
3.	Обращение к БД из прикладных программ
4.	Представление метаданных в прикладной программе.
5.	Разработка списков и справочников.
6.	Разработка MDI-приложений.
7.	Разработка интерфейсов фильтрации и сортировки, автоматическое конструирование запросов.
8.	Разработка интерфейса редактирование данных, карточки.
9.	Разработка OLAP-интерфейса.
10.	Введение в использование контроля версий.
11.	Разработка средств публикации отчётов.

\CourseControlSection

Для текущего контроля успеваемости используется устный опрос и проверка лабораторных работ.
Для оценок лабораторных работ используется 10-ти бальная шкала со следующими критериями (критерии и баллы по ним могут быть изменены в зависимости от конкретного задания):
Критерий	Баллы
Выполнение функциональных требований, правильное проектирование, использования и понимание средств языка программирования и технологий программирования, теоретическое обоснование. Культура и стиль кодирования.	8
Предоставление дополнительных функций или положительных особенностей, не предусмотренных заданием.	2
Итого	10

Для оценок устных ответов используется 10-ти бальная шкала со следующими критериями (критерии и баллы по ним могут быть изменены в зависимости от конкретной темы):
Критерий	Баллы
Правильный по сути ответ, умение обосновать его. Владение понятиями и терминами, способность использовать их при ответе на практические вопросы. Владение русским языком и культура речи. 	8
Изложение дополнительных материалов, сверх программы дисциплины. Подтверждение собственной правоты в случае несогласия с преподавателем. 	2
Итого	10

Аттестация по дисциплине – экзамен.
Оценка за освоение дисциплины определяется как оценка, включающая в себя оценки за лабораторные работы, тесты и экзамен.
В приложение к диплому вносится оценка за 4 семестр.

\CourseWorksSection

учебным планом не предусмотрены.

\CourseLiteratureSection

а) основная литература:
1.	Дейт К. Дж. Введение в системы баз данных. — 8-е изд. — М.: «Вильямс», 2006 — 1328 с
2.	Кузнецов С. Д. Основы баз данных. — 2-е изд. — М.: Интернет-Университет Информационных Технологий; БИНОМ. Лаборатория знаний, 2007. — 484 с.
3.	Гарсиа-Молина Г., Ульман Дж., Уидом Дж. Системы баз данных. Полный курс. — М.: Вильямс, 2003. — 1088 с.
4.	SQL:20nn Working Draft Documents, 2010
5.	Firebird Database Reference Manuals, 2000-2013
б) дополнительная литература
6.	Кренке Д. Теория и практика построения баз данных, 2005.
7.	Date C. J. An Introduction to Database Systems, 2004
8.	PostgreSQL Manuals, version 9.3, 2013
9.	MariaDB Manuals, 2012-2013
10.	Gulutzan Peter,  Pelz Trudy, SQL-99 Complete, Really, 2011
4.2. Электронные  ресурсы:
http://sql.ru
http://firebirdsql.org
http://flamerobin.org
http://mariadb.org
http://postgresql.org
http://stackoverflow.com/

\end{document}
