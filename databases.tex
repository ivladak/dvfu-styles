\documentclass[a4paper]{article}
\usepackage{dvfu}
\addbibresource{databases.bib}
\begin{document}

\renewcommand{\CourseName}{Базы данных}
\renewcommand{\CourseAuthor}{старший преподаватель кафедры ин\-фор\-ма\-тики,
  математического и компьютерного моделирования А.~С.~Кленин}
\renewcommand{\CourseYear}{1}
\renewcommand{\CourseSemester}{2}
\renewcommand{\CourseLectures}{18}
\renewcommand{\CourseLabs}{54}

\renewcommand{\CourseTotalAuditory}{72}
\renewcommand{\CourseHomework}{54}

\renewcommand{\CourseExamSemester}{2}

\CoursePlanTitlePage
\begin{CoursePlanTitlePageReverse}
  \CourseUpdateRecord
  \CourseUpdateRecord
\end{CoursePlanTitlePageReverse}

\CourseAnnotationSection

Дисциплина относится к обязательной части цикла математического и естественнонаучного цикла
основной образовательной программы подготовки бакалавров на направлении подготовки 010400.62 «Прикладная математика и информатика».

Дисциплина базируется на следующих дисциплинах: «Практикум на ЭВМ», «Основы информатики», «Основы информатики 2».

Знания, полученные по освоении дисциплины, необходимы при выполнении предквалификационной практики и выпускной квалификационной работы.

\CourseGoal

Дать студентам знания о теоретических основах СУБД, включая реляционную теорию, элементы теории распределённых систем,
а также умения настраивать и поддерживать СУБД.
Уделяется внимание глубокому изучению практических аспектов программирования прикладных приложений,
использующих реляционные СУБД, включая технологии ORM.

\CourseTasks
\begin{itemize}
  \item познакомить студентов с теоретическими основами СУБД;
  \item научить студентов устанавливать, настраивать и сопровождать СУБД;
  \item научить студентов проектировать БД;
  \item научить студентов разрабатывать прикладные программы, использующие БД.
\end{itemize}

В результате освоения учебной дисциплины у обучающихся должны обладать следующими компетенциями:

\begin{itemize}
  \Competences{CommonCompetences}{5&9&11&12&14&15}{;}
  \Competences{ProfessionalCompetences}{1&4&10}{.}
\end{itemize}

В результате освоения учебной дисциплины обучающиеся должны демонстрировать следующие результаты образования:

Знать о:
\begin{itemize}
  \item назначении, истории развития и основных возможностях СУБД;
  \item моделях БД, реляционной теории, проектировании БД;
  \item нормализации и нормальных формах БД;
  \item языке SQL;
  \item многопользовательском доступе, транзакциях и свойствах ACID;
  \item методах подключения к БД, много-звенной и клиент-серверной архитектуре;
  \item библиотеках языков программирования высокого уровня для подключения к СУБД, объектно-реля\-цион\-ных отображениях;
  \item аналитических средствах БД, OLAP и OLT;
  \item распределённых БД;
  \item управлении доступом и обеспечении информационной безопасности, администрировании и оптимизации БД.
\end{itemize}

Уметь:
\begin{itemize}
  \item строить реляционные модели предметной области;
  \item проектировать, производить нормализацию и денормализацию БД;
  \item устанавливать, настраивать и администрировать СУБД;
  \item разрабатывать прикладные приложения, работающие с БД.
\end{itemize}

\CourseTheorySection

\CourseSection{Основные сведения}

\CourseTopic{История и основные понятия СУБД}{2}{0}
\begin{itemize}
  \item Понятие базы данных и системы управления базами данных. Основные функции СУБД.
    Модели баз данных: неструктурированная, иерархическая, сетевая, реляционная.
  \item История развития СУБД. Обзор рынка СУБД.
  \item Основные коммерческие СУБД: Oracle, Microsoft SQL Server.
  \item Основные открытые реляционные СУБД: Maria DB, PostgreSQL, Firebird, SQLite.
  \item Основные открытые нереляционные СУБД..
\end{itemize}

\CourseTopic{Язык SQL}{3}{0}
\begin{itemize}
  \item Понятие языка запросов.
  \item Назначение и история развития языка SQL.
  \item Стандарты SQL, совместимость различных СУБД.
  \item Типы данных, операции и функции в SQL.
  \item Основные объекты СУБД: таблицы, поля, ограничения, домены, триггеры,
    хранимые процедуры и функции, внешние процедуры и функции, пользователи и роли.
  \item Понятие метаданных, Data Definition Language (DDL), операторы CREATE, DROP, ALTER.
  \item Data Manipulation Language (DML), операторы INSERT, UPDATE, DELETE.
  \item Оператор SELECT, предложения FROM, WHERE, ORDER BY, ROWS.
    Соединения, INNER JOIN, LEFT/RIGHT JOIN, FULL OUTER JOIN. Группировка, агрегатные функции, предложения GROUP BY, HAVING.
  \item Прочие языки запросов: XQuery / XPath, DMX, LINQ.
\end{itemize}

\CourseTopic{Разработка прикладных программ, работающих с БД}{3}{0}
\begin{itemize}
  \item Клиент-серверная архитектура, многозвенная архитектура. Протоколы СУБД, требования и особенности.
    Распределение функций между сервером СУБД и клиентской библиотекой СУБД.
    Совместимость СУБД, общие протоколы, ODBC, JDBC, стандарты Microsoft.
  \item Библиотеки доступа к БД из языков программирования высокого уровня. Middleware.
    Уровни абстрагирования библиотек: универсальное подключение, конструирование запросов,
    объект\-но-реля\-цион\-ное отображение (ORM), фреймворки. Достоинства и недостатки ORM, примеры библиотек ORM.
  \item Этапы обработки SQL-запроса. Параметры запросов. Атака SQL Injection и защита от неё.
    Оптимизация повторяющихся запросов с помощью предварительной подготовки.
\end{itemize}

\CourseTopic{Многопользовательский доступ}{2}{0}
\begin{itemize}
  \item Понятие целостности данных, виды нарушения целостности.
  \item Понятие транзакции, свойства ACID.
  \item Уровни изоляции, поддержка в различных СУБД,
    влияние уровня изоляции на производительность, критерии выбора уровня изоляции.
  \item Методы реализаций механизма транзакций, блокировка, версионирование. Обработка конфликтов.
\end{itemize}

\CourseTopic{Теория и проектирование реляционных БД}{2}{0}
\begin{itemize}
  \item Формальные понятия отношения, кортежа, элемента данных. Основные операции реляционной алгебры.
  \item Понятия зависимости, ключа, кандидата в ключи, суперключа, первичного и внешнего ключа, суррогатного ключа.
  \item Нормализация, нормальные формы, первая, вторая, третья, четвертая, пятая нормальные формы. Прочие нормальные формы.
  \item Преимущества и недостатки нормализации. Денормализация, причины и методы.
  \item Проектирование баз данных. Цели и задачи проектирования. Проектирование БД как инструмент анализа предметной области.
  \item Визуальное проектирование БД. Программные средства визуального проектирования. Языки проектирования БД: UML, IDEFx.
\end{itemize}

\CourseTopic{Аналитические возможности СУБД}{2}{0}
\begin{itemize}
  \item Аналитические возможности СУБД. Многомерная модель. Понятия OLAP и OLTP. Аналитические расширения языка SQL: CTE, оконные функции.
  \item Требования к пользовательскому интерфейсу OLAP-систем, библиотеки и инструментальные средства для аналитических запросов.
\end{itemize}

\CourseSection{Дополнительные сведения}

\CourseTopic{Распределённые базы данных}{2}{0}
\begin{itemize}
  \item Понятие распределённой базы, причины, преимущества и недостатки распределения, требования к распределённым БД.
    Проблемы реализации распределённых транзакций, теорема CAP.
  \item Механизмы синхронизации распределённых БД, двух-фазная блокировка (2PL, S2PL, SS2PL), двух-фазное завершение (2PC).
  \item Репликация, режимы репликации. Проблемы суррогатных ключей при двусторонней репликации, использование GUID.
\end{itemize}

\CourseTopic{Администрирование баз данных}{1}{0}
\begin{itemize}
  \item Задачи и инструменты администрирования. Контроль прав доступа, распределение ролей. Мониторинг БД.
  \item Модели рисков для БД, борьба с рисками. Резервное копирование.
\end{itemize}

\CourseTopic{Оптимизация баз данных}{1}{0}
\begin{itemize}
  \item Средства оценки и мониторинга производительности. Основные источники проблем с производительностью, принципы оптимизации.
  \item Оптимизация запросов. Индексы, виды индексов. План исполнения запросов, его анализ и изменение.
  \item Обзор алгоритмов и структур данных, используемых при реализации СУБД.
\end{itemize}

\CoursePracticeSection

\CoursePracticeTopic{Установка и настройка СУБД, установка клиентского приложения}{2}{0}
\CoursePracticeTopic{Построение SQL-запросов}{4}{0}
\CoursePracticeTopic{Обращение к БД из прикладных программ}{4}{0}
\CoursePracticeTopic{Представление метаданных в прикладной программе}{4}{0}
\CoursePracticeTopic{Разработка списков и справочников}{6}{0}
\CoursePracticeTopic{Разработка MDI-приложений}{4}{0}
\CoursePracticeTopic{Разработка интерфейсов фильтрации и сортировки, автоматическое конструирование запросов}{6}{0}
\CoursePracticeTopic{Разработка интерфейса редактирования данных, карточки}{8}{0}
\CoursePracticeTopic{Разработка OLAP-интерфейса}{8}{0}
\CoursePracticeTopic{Введение в использование контроля версий}{4}{0}
\CoursePracticeTopic{Разработка средств публикации отчётов}{4}{0}

\CourseControlSection

Для текущего контроля успеваемости используется устный опрос и проверка лабораторных работ.
Для оценок лабораторных работ используется 10-ти бальная шкала со следующими критериями
(критерии и баллы по ним могут быть изменены в зависимости от конкретного задания):

\begin{CourseMarkCriteria}
  \CourseMarkCriterium{8}{Выполнение функциональных требований. Правильное проектирование.
    Использования и понимание средств языка программирования и технологий программирования.
    Теоретическое обоснование решения. Культура и стиль кодирования.}
  \CourseMarkCriterium{2}{Реализация дополнительных функций или положительных особенностей,
    не предусмотренных заданием.}
\end{CourseMarkCriteria}

Для оценок устных ответов используется 10-ти бальная шкала со следующими критериями
(критерии и баллы по ним могут быть изменены в зависимости от конкретной темы):

\begin{CourseMarkCriteria}
  \CourseMarkCriterium{8}{Правильный по сути ответ, умение обосновать его.
    Владение понятиями и терминами, способность использовать их при ответе на практические вопросы.
    Владение русским языком и культура речи.}
  \CourseMarkCriterium{2}{Изложение дополнительного материала, сверх программы дисциплины.
    Подтверждение собственной правоты в случае несогласия с преподавателем.}
\end{CourseMarkCriteria}

Аттестация по дисциплине~— экзамен.
Оценка за освоение дисциплины определяется как оценка, включающая в себя оценки за лабораторные работы, тесты и экзамен.
В приложение к диплому вносится оценка за \CourseExamSemester\ семестр.

\begin{CourseControlQuestions}{экзамену}
\item История развития СУБД. Назначение, виды и основные возможности СУБД.
\item Реляционная модель данных. Основные операции, свойства и преимущества.
\item Логическая модель данных. Нормализация и денормализация. Нормальные формы.
\item Клиент-серверная архитектура. Двух- и трёхзвенная архитектура.
\item Многопользовательский доступ, транзакции, ACID.
\item Работа с СУБД из прикладных программ. Middleware, ORM.
\item Язык SQL. История развития и основные конструкции DML.
\item Язык SQL. Управление правами доступа и основные конструкции DDL.
\item Язык SQL. Средства конструирования сложных запросов.
\item Язык SQL. Хранимые и внешние процедуры. Триггеры.
\item Распределённые БД. Репликация. Резервное копирование.
\item Режимы OLTP и OLAP. Многомерная модель. Средства построения OLAP-отчётов.
\item Производительность БД. Факторы, методы настройки. Виды индексов.
\item Визуальное проектирование БД. Языки и стандарты.
\end{CourseControlQuestions}

\CourseWorksSection

Учебным планом не предусмотрены.

\CourseLiteratureSection

\end{document}
